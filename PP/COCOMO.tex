\documentclass{article}
\usepackage{graphicx}
\usepackage[fontsize=11pt]{scrextend}
\title{PowerEnjoy Service - Project Plan}
\usepackage{float}
\begin{document}
	
	\subsection{Cost and Effort Estimation: COCOMO II}
	In this section we are going to use the COCOMO II method to estimate the cost and effort which would be needed to development this application -- PowerEnjoy.
	
	\subsubsection{Scale Drivers}
	In order to evaluate the cost and effort which should be applied in this project, we refer to the official COCOMO II table which is released:
	\begin{table}[H]
		\centering
		\caption{Scale Factor values, SFj, for COCOMO II Models}
		\label{my-label}
		\begin{tabular}{@{}|l|l|l|l|l|l|l|@{}}
		\hline
		Scale Factors     
		& Very Low                                                                                                                                & Low                                                                              & Normal                                                                                       & High                                                                    & Very High                                                           & \begin{tabular}[c]{@{}l@{}}Extra\\ High\end{tabular}                     \\ 
		\hline
		PREC,SFj                                           & \begin{tabular}[c]{@{}l@{}}thoroughly \\ unprece-\\ dented\\ 6.20\end{tabular} & \begin{tabular}[c]{@{}l@{}}largely\\ unprece-\\ dented\\ 4.96\end{tabular}       & \begin{tabular}[c]{@{}l@{}}somewhat\\ unprece-\\ dented\\ 3.72\end{tabular}                   & \begin{tabular}[c]{@{}l@{}}generally\\ familiar\\2.48\end{tabular}       & \begin{tabular}[c]{@{}l@{}}largely fa-\\ miliar\\1.24\end{tabular}   & \begin{tabular}[c]{@{}l@{}}thoroughly\\ familiar\\0.00\end{tabular}       \\ 
		\hline
		FLEX,SFj                                           & rigorous 5.07                                                                    & \begin{tabular}[c]{@{}l@{}}occasional\\ relaxation\\4.05\end{tabular}             & \begin{tabular}[c]{@{}l@{}}some\\ relaxation\\3.04\end{tabular}                                & \begin{tabular}[c]{@{}l@{}}general\\ confor-\\ mity\\ 2.03\end{tabular} & \begin{tabular}[c]{@{}l@{}}some con-\\ formity\\1.01\end{tabular}    & \begin{tabular}[c]{@{}l@{}}general\\ goals\\0.00\end{tabular}             \\
		\hline
		\begin{tabular}[c]{@{}l@{}}
		RESL\\ SFj\end{tabular} & \begin{tabular}[c]{@{}l@{}}little\\ (20\%)\\ 7.07\end{tabular}                   & \begin{tabular}[c]{@{}l@{}}some\\ (40\%)\\ 5.65\end{tabular}                     & \begin{tabular}[c]{@{}l@{}}often\\ (60\%)\\ 4.24\end{tabular}                                 & \begin{tabular}[c]{@{}l@{}}generally\\ (75\%)\\ 2.83\end{tabular}       & \begin{tabular}[c]{@{}l@{}}mostly\\ (90\%)\\ 1.41\end{tabular}      & \begin{tabular}[c]{@{}l@{}}full\\ (100\%)\\ 0.00\end{tabular}            \\ 
		\hline
		TEAM,SFj                                           & \begin{tabular}[c]{@{}l@{}}very diffi-\\ cult inter-\\ actions\\5.48\end{tabular} & \begin{tabular}[c]{@{}l@{}}some diffi-\\ cult inter-\\ actions\\4.38\end{tabular} & \begin{tabular}[c]{@{}l@{}}basically\\ coop-\\ erative\\ interac-\\ tions\\ 3.29\end{tabular} & \begin{tabular}[c]{@{}l@{}}largely co-\\ operative\\2.19\end{tabular}    & \begin{tabular}[c]{@{}l@{}}highly co-\\ operative\\1.10\end{tabular} & \begin{tabular}[c]{@{}l@{}}seamless\\ interac-\\ tions\\0.00\end{tabular} \\ 
		\hline
		\begin{tabular}[c]{@{}l@{}}
		PMAT\\ SFj\end{tabular} & \begin{tabular}[c]{@{}l@{}}Level 1\\ Lower\\ 7.80\end{tabular}                   & \begin{tabular}[c]{@{}l@{}}Level 1\\ Upper\\ 6.24\end{tabular}                   & \begin{tabular}[c]{@{}l@{}}Level 2\\ 4.68\end{tabular}                                        & \begin{tabular}[c]{@{}l@{}}Level 3\\ 3.12\end{tabular}                  & \begin{tabular}[c]{@{}l@{}}Level 4\\ 1.56\end{tabular}              & \begin{tabular}[c]{@{}l@{}}Level 5 \\ 0.00\end{tabular} \\ 
		\hline
		\end{tabular}
	\end{table}
	A brief description for each scale driver:
	\begin{itemize}
		\item Precedentedness: Precedentedness would be high if the project is similar to the previous developed projects. So the Precedentedness would be depended on the experience of out team with the development of this kind of project. Since this is the first time for our team members to manage and develop such a big project, this value should be Low.
		\item Development Flexibility: Development Flexibility would be high if there are no specific constraints to conform to pre-established requirements and external interface specs. Since in this project, there are strict requirements, but without limitation for the implementation method. This value should be Normal.
		\item Risk Resolution: Risk Resolution should be high if we have a good risk management plan, clear definition of budget and schedule, focus on architectural definition. As the result of our analysis, we have a great and extensive risk analysis. Therefore, this value should be high.
		\item Team Cohesion: Team Cohesion should be high if all stakeholders are able to work in a team and share the same vision and commitment. Since the members in our team live in the same city and we know each other perfectly, we can work in a cooperation way. therefore, this value should be very high.
		\item Process Maturity: Process Maturity refers to a well known method for assessing the maturity of a software organization, CMM, now evolved into CMMI. Although we do not have experience about the development of such a big project, we have achieved all the requirements successfully. And we also have some experience about the Java projects, so this value should be set to Normal.
	\end{itemize}
	Overall, the result of our assessment is as follow:
	\begin{table}[H]
		\centering
		\caption{Result of Scale Drivers}
		\label{my-label}
		\begin{tabular}{|l|l|l|}
			\hline
			Scale Driver                       & Factor    & Value \\ \hline
			Precedentedness (PREC)   & Low       & 4.96   \\ \hline
			Development flexibility (FLEX)  & Normal       & 3.04   \\ \hline
			Risk resolution (RESL) & High       & 2.83   \\ \hline
			Team cohesion (TEAM)  & Very High      & 1.10   \\ \hline
			Process maturity (PMAT)      & Normal    & 4.68 \\ \hline
			\multicolumn{2}{|l|}{Total}                    & 16.61  \\ \hline
		\end{tabular}
	\end{table}
	\newpage

	\subsubsection{Cost Drivers}
	There are 17 Cost Drivers for the Post-Architecture:
	\begin{itemize}
		\item Required Software Reliability (RELY)
		\item Database size (DATA)
		\item Product complexity (CPLX)
		\item Required reusability (RUSE)
		\item Documentation match to life-cycle needs (DOCU)
		\item Execution time constraint (TIME)
		\item Storage constraint (STOR)
		\item Platform Volatility (PVOL)
		\item Analyst Capability (ACAP)
		\item Programmer Capability (PCAP)
		\item Application Experience (APEX)
		\item Platform Experience (PLEX)
		\item Language and Tool Experience (LTEX)
		\item Personnel continuity (PCON)
		\item Usage of Software Tools (TOOL)
		\item Multisite development (SITE)
		\item Required development schedule (SCED)
	\end{itemize}
	\newpage
	
	We have analysed the Cost Drivers step by step:
	
	\begin{itemize}
		
		\item Required Software Reliability (RELY):\\
		Since the PowerEnjoy is the only way for the user to get the services, this system should be reliable. Otherwise there would be financial loss of the company, and would lead to inconveniences for the users. Therefore, RELY should be High.
		\begin{table}[H]
			\centering
			\caption{RELY Cost Drivers}
			\label{my-label}
			\begin{tabular}{@{}|l|l|l|l|l|l|l|@{}}
				\hline
				RELY descriptors                                               & \begin{tabular}[c]{@{}l@{}}slightly\\ inconve-\\ nience\end{tabular} & \begin{tabular}[c]{@{}l@{}}easily re-\\ coverable\\ losses\end{tabular} & \begin{tabular}[c]{@{}l@{}}moderate\\ recov-\\ erable losses\end{tabular} & \begin{tabular}[c]{@{}l@{}}high \\ financial \\ loss\end{tabular} & \begin{tabular}[c]{@{}l@{}}risk to hu-\\ man life\end{tabular} &            \\
				\hline
				Rating level                                                   & Very low                                                             & Low                                                                     & Normal                                                                    & High                                                              & Very High                                                      & Extra High \\ 
				\hline
				\begin{tabular}[c]{@{}l@{}}
				Effort mul-\\ tipliers\end{tabular} & 0.82                                                                 & 0.92                                                                    & 1.00                                                                      & 1.10                                                              & 1.26                                                           & n / a      \\ 
				\hline
			\end{tabular}
		\end{table}
		
		\item Database size (DATA):\\
		This measure considers the effective size of our database. In fact, we have no way to get the extremely precise answer. We can only estimate the Database Size roughly. Since we have estimated the SLOC = 12464, and we set the ratio D/P to be 500. So the DATA Cost Drivers should be High.
	\begin{table}[H]
		\centering
		\caption{DATA Cost Drivers}
		\label{my-label}
		\begin{tabular}{|l|l|l|l|l|l|l|}
			\hline
			\begin{tabular}[c]{@{}l@{}}DATA De-\\ scriptors\end{tabular} &  & \begin{tabular}[c]{@{}l@{}}D/P \\ <= 10\end{tabular} & \begin{tabular}[c]{@{}l@{}}10 <= \\ D/P \\ <= 100\end{tabular} & \begin{tabular}[c]{@{}l@{}}100 <= \\ D/P \\ <= 1000\end{tabular} & \begin{tabular}[c]{@{}l@{}}D/P \\ >= \\ 1000\end{tabular} &  \\ \hline
			Rating level & Very Low & Low & Nonimal & High & Very High & Extra High \\ \hline
			\begin{tabular}[c]{@{}l@{}}Effort mul-\\ tipliers\end{tabular} & n/a & 0.90 & 1.00 & 1.14 & 1.28 & n/a \\ \hline
		\end{tabular}
	\end{table}
				
		
		\item Product complexity (CPLX):\\
		Set to High, due to the SLOC is large.
		\begin{table}[H]
			\centering
			\caption{CPLX Cost Driver}
			\label{my-label}
			\begin{tabular}{|l|l|l|l|l|l|l|}
				\hline
				Rating level & Very low & Low & Nominal & High & Very High & \begin{tabular}[c]{@{}l@{}}Extra\\ High\end{tabular} \\ \hline
				\begin{tabular}[c]{@{}l@{}}Effort mul-\\ tipliers\end{tabular} & 0.73 & 0.87 & 1.00 & 1.17 & 1.34 & 1.74 \\ \hline
			\end{tabular}
		\end{table}
		
		
		\item Required reusability (RUSE):\\
		In this project, the codes and documents would only be used by the this project itself. Therefore, the RUSE should be set to Nominal.
		\begin{table}[H]
			\centering
			\caption{RUSE Cost Driver}
			\label{my-label}
			\begin{tabular}{|l|l|l|l|l|l|l|}
				\hline
				\begin{tabular}[c]{@{}l@{}}RUSE De-\\ scriptors\end{tabular} &  & None & \begin{tabular}[c]{@{}l@{}}Across\\ project\end{tabular} & \begin{tabular}[c]{@{}l@{}}Across \\ program\end{tabular} & \begin{tabular}[c]{@{}l@{}}Across\\ product\\ line\end{tabular} & \begin{tabular}[c]{@{}l@{}}Across\\ multiple\\ product\\ lines\end{tabular} \\ \hline
				Rating level & Very Low & Low & Nominal & High & Very High & \begin{tabular}[c]{@{}l@{}}Extra\\ High\end{tabular} \\ \hline
				\begin{tabular}[c]{@{}l@{}}Effort mul-\\ tipliers\end{tabular} & n/a & 0.95 & 1.00 & 1.07 & 1.15 & 1.24 \\ \hline
			\end{tabular}
		\end{table}
		
		
		\item Documentation match to life-cycle needs (DOCU):\\
		This value depends on the relationship between the documents and the requirements. In our project, we have satisfied every requirements for the application. Therefore, this value should be set to High.
		\begin{table}[H]
			\centering
			\caption{DOCU Cost Driver}
			\label{my-label}
			\begin{tabular}{|l|l|l|l|l|l|l|}
				\hline
				\begin{tabular}[c]{@{}l@{}}DOCU De-\\ scriptors\end{tabular} & \begin{tabular}[c]{@{}l@{}}Many\\ life-cycle\\ needs\\ uncovered\end{tabular} & \begin{tabular}[c]{@{}l@{}}Some\\ life-cycle\\ needs\\ uncovered\end{tabular} & \begin{tabular}[c]{@{}l@{}}Right-\\ sized to\\ life-cycle\\ needs\end{tabular} & \begin{tabular}[c]{@{}l@{}}Excessive\\ for life-\\ cycle\\ needs\end{tabular} & \begin{tabular}[c]{@{}l@{}}Very ex-\\ cessive for\\ life-cycle\\ needs\end{tabular} &  \\ \hline
				Rating level & Very Low & Low & Nominal & High & Very High & \begin{tabular}[c]{@{}l@{}}Extra\\ High\end{tabular} \\ \hline
				\begin{tabular}[c]{@{}l@{}}Effort mul-\\ tipliers\end{tabular} & 0.81 & 0.91 & 1.00 & 1.11 & 1.23 & n/a \\ \hline
			\end{tabular}
		\end{table}
		
		\item Execution time constraint (TIME):\\
		This value depends on the expected usage of CPU when the software is working. Since this application should response rapidly, we suppose the TIME should be set to Very High.
		\begin{table}[H]
			\centering
			\caption{TIME Cost Driver}
			\label{my-label}
			\begin{tabular}{|l|l|l|l|l|l|l|}
				\hline
				\begin{tabular}[c]{@{}l@{}}TIME De-\\ scriptors\end{tabular} &  &  & \begin{tabular}[c]{@{}l@{}}<=50\%\\ use of\\ available\\ execution\\ time\end{tabular} & \begin{tabular}[c]{@{}l@{}}70\% use of\\ available\\ execution\\ time\end{tabular} & \begin{tabular}[c]{@{}l@{}}85\% use of\\ available\\ execution\\ time\end{tabular} & \begin{tabular}[c]{@{}l@{}}95\% use of\\ available\\ execution\\ time\end{tabular} \\ \hline
				Rating level & Very Low & Low & Nominal & High & Very High & \begin{tabular}[c]{@{}l@{}}Extra\\ High\end{tabular} \\ \hline
				\begin{tabular}[c]{@{}l@{}}Effort mul-\\ tipliers\end{tabular} & n/a & n/a & 1.00 & 1.11 & 1.29 & 1.63 \\ \hline
				\end{tabular}
			\end{table}
				
		
		\item Storage constraint (STOR):\\
		This value depends on the capability of storage of the hardware when the software is working. Since nowadays the capability of disk drivers can easily reach a high level, and the cost of such kinds of disk drivers would be cheap. Therefore, this value should be set to Nominal.
		\begin{table}[H]
			\centering
			\caption{STOR Cost Driver}
			\label{my-label}
			\begin{tabular}{|l|l|l|l|l|l|l|}
				\hline
				\begin{tabular}[c]{@{}l@{}}STOR De-\\ scriptors\end{tabular} &  &  & \begin{tabular}[c]{@{}l@{}}≤ 50\%\\ use of\\ available\\ storage\end{tabular} & \begin{tabular}[c]{@{}l@{}}70\% use of\\ available\\ storage\end{tabular} & \begin{tabular}[c]{@{}l@{}}85\% use of\\ available\\ storage\end{tabular} & \begin{tabular}[c]{@{}l@{}}95\% use of\\ available\\ storage\end{tabular} \\ \hline
					Rating level & Very Low & Low & Nominal & High & Very High & \begin{tabular}[c]{@{}l@{}}Extra\\ High\end{tabular} \\ \hline
					\begin{tabular}[c]{@{}l@{}}Effort mul-\\ tipliers\end{tabular} & n/a & n/a & 1.00 & 1.05 & 1.17 & 1.46 \\ \hline
				\end{tabular}
			\end{table}
		
		
		\item Platform Volatility (PVOL):\\
		In fact, we do not expect the version of application changes so often. But the user application may require some new version for satisfy the change of  mobile-phone operating system. what's more, some user may want to have some new functions. Therefore, this system may have to be release twice a year. Overall, this value should be set to Nominal.
		\begin{table}[H]
			\centering
			\caption{PVOL Cost Driver}
			\label{my-label}
			\begin{tabular}{|l|l|l|l|l|l|l|}
				\hline
				\begin{tabular}[c]{@{}l@{}}PVOL De-\\ scriptors\end{tabular} &  & \begin{tabular}[c]{@{}l@{}}Major\\ change\\ every\\ 12 mo.,\\ minor\\ change\\ every 1\\ mo.\end{tabular} & \begin{tabular}[c]{@{}l@{}}Major:\\ 6mo;\\ minor:\\ 2wk.\end{tabular} & \begin{tabular}[c]{@{}l@{}}Major:\\ 2mo,\\ minor:\\ 1wk\end{tabular} & \begin{tabular}[c]{@{}l@{}}Major:\\ 2wk; mi-\\ nor: 2\\ days\end{tabular} &  \\ \hline
				Rating level & Very Low & Low & Nominal & High & Very High & \begin{tabular}[c]{@{}l@{}}Extra\\ High\end{tabular} \\ \hline
				\begin{tabular}[c]{@{}l@{}}Effort mul-\\ tipliers\end{tabular} & n/a & 0.87 & 1.00 & 1.15 & 1.30 & n/a \\ \hline
			\end{tabular}
		\end{table}
		
		
		\item Analyst Capability (ACAP):\\
		We think we have finished analysis documents appropriately. For this reason, this value should be set to High.
		\begin{table}[H]
			\centering
			\caption{ACAP Cost Driver}
			\label{my-label}
			\begin{tabular}{|l|l|l|l|l|l|l|}
				\hline
				\begin{tabular}[c]{@{}l@{}}ACAP De-\\ scriptors\end{tabular} & \begin{tabular}[c]{@{}l@{}}15th per-\\ centile\end{tabular} & \begin{tabular}[c]{@{}l@{}}35th per-\\ centile\end{tabular} & \begin{tabular}[c]{@{}l@{}}55th per-\\ centile\end{tabular} & \begin{tabular}[c]{@{}l@{}}75th per-\\ centile\end{tabular} & \begin{tabular}[c]{@{}l@{}}90th per-\\ centile\end{tabular} &  \\ \hline
				Rating level & Very Low & Low & Nominal & High & Very High & \begin{tabular}[c]{@{}l@{}}Extra\\ High\end{tabular} \\ \hline
				\begin{tabular}[c]{@{}l@{}}Effort mul-\\ tipliers\end{tabular} & 1.42 & 1.19 & 1.00 & 0.85 & 0.71 & n/a \\ \hline
			\end{tabular}
		\end{table}
		
		
		\item Programmer Capability (PCAP):\\
		We have no way to get a extremely precise result for this value, since we would not finish the implementation part. But we can estimate roughly. Since we have finished some Java program, this value should be set to Nominal.
		\begin{table}[H]
			\centering
			\caption{PCAP Cost Driver}
			\label{my-label}
			\begin{tabular}{|l|l|l|l|l|l|l|}
				\hline
				\begin{tabular}[c]{@{}l@{}}PCAP De-\\ scriptors\end{tabular} & \begin{tabular}[c]{@{}l@{}}15th per-\\ centile\end{tabular} & \begin{tabular}[c]{@{}l@{}}35th per-\\ centile\end{tabular} & \begin{tabular}[c]{@{}l@{}}55th per-\\ centile\end{tabular} & \begin{tabular}[c]{@{}l@{}}75th per-\\ centile\end{tabular} & \begin{tabular}[c]{@{}l@{}}90th per-\\ centile\end{tabular} &  \\ \hline
				Rating level & Very Low & Low & Nominal & High & Very High & \begin{tabular}[c]{@{}l@{}}Extra\\ High\end{tabular} \\ \hline
				\begin{tabular}[c]{@{}l@{}}Effort mul-\\ tipliers\end{tabular} & 1.35 & 1.15 & 1.00 & 0.88 & 0.76 & n/a \\ \hline
			\end{tabular}
		\end{table}
		
		
		\item Application Experience (APEX):\\
		We have not experiences about the implementation of  J2E project. We only have the experiences about JAVA implementation. Therefore, this value should be set to Low.
		\begin{table}[H]
			\centering
			\caption{APEX Cost Driver}
			\label{my-label}
			\begin{tabular}{|l|l|l|l|l|l|l|}
				\hline
				\begin{tabular}[c]{@{}l@{}}APEX De-\\ scriptors\end{tabular} & \begin{tabular}[c]{@{}l@{}}<= 2 \\ months\end{tabular} & \begin{tabular}[c]{@{}l@{}}6\\ months\end{tabular} & \begin{tabular}[c]{@{}l@{}}1\\ year\end{tabular} & \begin{tabular}[c]{@{}l@{}}3\\ year\end{tabular} & \begin{tabular}[c]{@{}l@{}}6\\ year\end{tabular} &  \\ \hline
				Rating level & Very Low & Low & Nominal & High & Very High & \begin{tabular}[c]{@{}l@{}}Extra\\ High\end{tabular} \\ \hline
				\begin{tabular}[c]{@{}l@{}}Effort mul-\\ tipliers\end{tabular} & 1.22 & 1.10 & 1.00 & 0.88 & 0.81 & n/a \\ \hline
			\end{tabular}
		\end{table}
		
		
		\item Platform Experience (PLEX):\\
		We have no experiences about the J2E implementation. But we have experiences about the Database and the Java. Therefore, we set this value to Nominal.
		\begin{table}[H]
			\centering
			\caption{PLEX Cost Driver}
			\label{my-label}
			\begin{tabular}{|l|l|l|l|l|l|l|}
				\hline
				\begin{tabular}[c]{@{}l@{}}PLEX De-\\ scriptors\end{tabular} & \begin{tabular}[c]{@{}l@{}}<= 2 \\ months\end{tabular} & \begin{tabular}[c]{@{}l@{}}6\\ months\end{tabular} & \begin{tabular}[c]{@{}l@{}}1\\ year\end{tabular} & \begin{tabular}[c]{@{}l@{}}3\\ year\end{tabular} & \begin{tabular}[c]{@{}l@{}}6\\ year\end{tabular} &  \\ \hline
				Rating level & Very Low & Low & Nominal & High & Very High & \begin{tabular}[c]{@{}l@{}}Extra\\ High\end{tabular} \\ \hline
				\begin{tabular}[c]{@{}l@{}}Effort mul-\\ tipliers\end{tabular} & 1.19 & 1.09 & 1.00 & 0.91 & 0.85 & n/a \\ \hline
			\end{tabular}
		\end{table}
		
		
		\item Language and Tool Experience (LTEX):\\
		We have no experiences about the J2E implementation. But we have experiences about the Database and the Java. Therefore, we set this value to Nominal.
		\begin{table}[H]
			\centering
			\caption{LTEX Cost Driver}
			\label{my-label}
			\begin{tabular}{|l|l|l|l|l|l|l|}
				\hline
				\begin{tabular}[c]{@{}l@{}}LTEX De-\\ scriptors\end{tabular} & \begin{tabular}[c]{@{}l@{}}<= 2 \\ months\end{tabular} & \begin{tabular}[c]{@{}l@{}}6\\ months\end{tabular} & \begin{tabular}[c]{@{}l@{}}1\\ year\end{tabular} & \begin{tabular}[c]{@{}l@{}}3\\ year\end{tabular} & \begin{tabular}[c]{@{}l@{}}6\\ year\end{tabular} &  \\ \hline
				Rating level & Very Low & Low & Nominal & High & Very High & \begin{tabular}[c]{@{}l@{}}Extra\\ High\end{tabular} \\ \hline
				\begin{tabular}[c]{@{}l@{}}Effort mul-\\ tipliers\end{tabular} & 1.20 & 1.09 & 1.00 & 0.91 & 0.84 & n/a \\ \hline
			\end{tabular}
		\end{table}
		
		
		\item Personnel continuity (PCON):\\
		Since the time we can spend on this project is quite limited. This value we should set to Very Low.
		\begin{table}[H]
			\centering
			\caption{PCON Cost Driver}
			\label{my-label}
			\begin{tabular}{|l|l|l|l|l|l|l|}
				\hline
				\begin{tabular}[c]{@{}l@{}}PCON De-\\ scriptors\end{tabular} & \begin{tabular}[c]{@{}l@{}}48\% /\\ year\end{tabular} & \begin{tabular}[c]{@{}l@{}}24\% /\\ year\end{tabular} & \begin{tabular}[c]{@{}l@{}}12\% /\\ year\end{tabular} & \begin{tabular}[c]{@{}l@{}}6\% /\\ year\end{tabular} & \begin{tabular}[c]{@{}l@{}}3\% /\\ year\end{tabular} &  \\ \hline
					Rating level & Very Low & Low & Nominal & High & Very High & \begin{tabular}[c]{@{}l@{}}Extra\\ High\end{tabular} \\ \hline
					\begin{tabular}[c]{@{}l@{}}Effort mul-\\ tipliers\end{tabular} & 1.29 & 1.12 & 1.00 & 0.90 & 0.81 & n/a \\ \hline
				\end{tabular}
			\end{table}
		
		
		\item Usage of Software Tools (TOOL):\\
		Since we have a very good application implementation environment, we should set this value to High
		\begin{table}[H]
			\centering
			\caption{TOOL Cost Driver}
			\label{my-label}
			\begin{tabular}{|l|l|l|l|l|l|l|}
				\hline
				\begin{tabular}[c]{@{}l@{}}TOOL De-\\ scriptors\end{tabular} & \begin{tabular}[c]{@{}l@{}}edit, code,\\ debug\end{tabular} & \begin{tabular}[c]{@{}l@{}}simple,\\ frontend,\\ backend\\ CASE,\\ little inte-\\ gration\end{tabular} & \begin{tabular}[c]{@{}l@{}}basic\\ life-cycle\\ tools,\\ mod-\\ erately\\ integrated\end{tabular} & \begin{tabular}[c]{@{}l@{}}strong,\\ mature\\ life-cycle\\ tools,\\ mod-\\ erately\\ integrated\end{tabular} & \begin{tabular}[c]{@{}l@{}}strong,\\ mature,\\ proactive\\ life-cycle\\ tools, well\\ integrated\\ with pro-\\ cesses,\\ methods,\\ reuse\end{tabular} &  \\ \hline
				Rating level & Very Low & Low & Nominal & High & Very High & \begin{tabular}[c]{@{}l@{}}Extra\\ High\end{tabular} \\ \hline
				\begin{tabular}[c]{@{}l@{}}Effort mul-\\ tipliers\end{tabular} & 1.17 & 1.09 & 1.00 & 0.90 & 0.78 & n/a \\ \hline
			\end{tabular}
		\end{table}
		
		
		\item Multisite development (SITE):\\
		The members in our team are live in the same city, and also thanks to the wideband Internet services, we can communicate with each other. Therefore, this value should be set to Very High.
		\begin{table}[H]
			\centering
			\caption{SITE Cost Driver}
			\label{my-label}
			\begin{tabular}{|l|l|l|l|l|l|l|}
				\hline
				\begin{tabular}[c]{@{}l@{}}SITE Collo-\\ cation De-\\ scriptors,\\ SITE \\ Com-\\ munications\\ Descriptors\end{tabular} & \begin{tabular}[c]{@{}l@{}}Intern-\\ ational,\\ Some\\ phone,\\ mail\end{tabular} & \begin{tabular}[c]{@{}l@{}}Multi-city\\ and multi-\\ company,\\ Individual\\ phone, fax\end{tabular} & \begin{tabular}[c]{@{}l@{}}Multi-city\\ or multi-\\ company,\\ Narrow\\ band\\ email\end{tabular} & \begin{tabular}[c]{@{}l@{}}Same city\\ or metro\\ area,Wideband\\ electronic\\ communi-\\ cation\end{tabular} & \begin{tabular}[c]{@{}l@{}}Same\\ build-\\ ing or\\ complex\\ Wideband\\ elect.\\ comm.,\\ occasional\\ video\\ conf.\end{tabular} & \begin{tabular}[c]{@{}l@{}}Fully col-\\ located,\\ Interactive\\ multime-\\ dia\end{tabular} \\ \hline
				Rating level & Very Low & Low & Nominal & High & Very High & \begin{tabular}[c]{@{}l@{}}Extra\\ High\end{tabular} \\ \hline
				\begin{tabular}[c]{@{}l@{}}Effort mul-\\ tipliers\end{tabular} & 1.22 & 1.09 & 1.00 & 0.93 & 0.86 & 0.80 \\ \hline
			\end{tabular}
		\end{table}
		
		
		\item Required development schedule (SCED):\\
		Although our available time in this project is limited, we have worked on this project in a consistent time. Therefore, this value should be Nominal.
		\begin{table}[H]
			\centering
			\caption{SCED Cost Driver}
			\label{my-label}
			\begin{tabular}{|l|l|l|l|l|l|l|}
				\hline
				\begin{tabular}[c]{@{}l@{}}SCED De-\\ scriptors\end{tabular} & \begin{tabular}[c]{@{}l@{}}75\%\\ nominal\end{tabular} & \begin{tabular}[c]{@{}l@{}}85\%\\ nominal\end{tabular} & \begin{tabular}[c]{@{}l@{}}100\%\\ nominal\end{tabular} & \begin{tabular}[c]{@{}l@{}}130\%\\ nominal\end{tabular} & \begin{tabular}[c]{@{}l@{}}160\%\\ nominal\end{tabular} &  \\ \hline
					Rating level & Very Low & Low & Nominal & High & Very High & \begin{tabular}[c]{@{}l@{}}Extra\\ High\end{tabular} \\ \hline
					\begin{tabular}[c]{@{}l@{}}Effort mul-\\ tipliers\end{tabular} & 1.43 & 1.14 & 1.00 & 1.00 & 1.00 & n/a \\ \hline
				\end{tabular}
			\end{table}	
	\end{itemize}
	Overall, our results are as follows:
		\begin{table}[H]
			\centering
			\caption{Result of Cost Drivers}
			\label{my-label}
			\begin{tabular}{|l|l|l|}
				\hline
				Cost Driver & Factor & Value \\ \hline
				Required Software Reliability (RELY) & High & 1.10 \\ \hline
				Database size (DATA) & High & 1.14 \\ \hline
				Product complexity (CPLX) & High & 1.17 \\ \hline
				Required Reusability (RUSE) & Nominal & 1.00 \\ \hline
				Documentation match to life-cycle needs (DOCU) & High & 1.11 \\ \hline
				Execution Time Constraint (TIME) & Very High & 1.29 \\ \hline
				Main storage constraint (STOR) & Nominal & 1.00 \\ \hline
				Platform volatility (PVOL) & Nominal & 1.00 \\ \hline
				Analyst capability (ACAP) & High & 0.85 \\ \hline
				Programmer capability (PCAP) & Nominal & 1.00 \\ \hline
				Application Experience (APEX) & Low & 1.10 \\ \hline
				Platform Experience (PLEX) & Nominal & 1.00 \\ \hline
				Language and Tool Experience (LTEX) & Nominal & 1.00 \\ \hline
				Personnel continuity (PCON) & Very Low & 1.29 \\ \hline
				Usage of Software Tools (TOOL) & High & 0.90 \\ \hline
				Multisite development (SITE) & Very High & 0.86 \\ \hline
				Required development schedule (SCED) & Nominal & 1.00 \\ \hline
				\multicolumn{2}{|l|}{Total} & 1.96127 \\ \hline
			\end{tabular}
		\end{table}
	
	\newpage
	
	\subsubsection{Effort Equation}
	This final equation gives us the effort estimation measured in Person-Months (PM):
	\begin{table}[H]
		\centering
		\label{my-label}
		\begin{tabular}{|l|}
			\hline
			Effort = A ∗ EAF ∗ KSLOCE \\ \hline
		\end{tabular}
	\end{table}
	\begin{table}[H]
		\centering
		\label{my-label}
		\begin{tabular}{|l|}
			\hline
			\begin{tabular}[c]{@{}l@{}}A = 2.94 (for COCOMO II)\\ EAF=product of all cost drivers (1.96127)\\ E = exponent derived from the scale drivers. It is computed a,s:\\ B+ 0.01 ∗,SF[i]=B+ 0.01 ∗ 16.61 = 0.91 + 0.1661 =,1.0761\\ in which B is equal to: 0.91 for COCOMO II.\end{tabular} \\ \hline
		\end{tabular}
	\end{table}

	\begin{table}[H]
		\centering
		\label{my-label}
		\begin{tabular}{|l|}
			\hline
			\begin{tabular}[c]{@{}l@{}}Effort = A ∗ EAF ∗ KSLOC\^E = 2.94 ∗ 1.96127 ∗ 12.464 \^ 1.0761 =\\ 87.081 PM ≈ 87 PM\end{tabular} \\ \hline
		\end{tabular}
	\end{table}
	
	\newpage
	\subsubsection{Schedule Estimation}
	Regarding the final schedule, we are going to use the following formula:
	\begin{table}[H]
		\centering
		\label{my-label}
		\begin{tabular}{|l|}
			\hline
			Duration = 3.67 ∗ Effort \^ F \\ \hline
		\end{tabular}
	\end{table}
	\begin{table}[H]
		\centering
		\label{my-label}
		\begin{tabular}{|l|}
			\hline
			\begin{tabular}[c]{@{}l@{}}F = 0.28 + 0.2 * (E-B) = 0.28 + 0.2 * (1.0761 - 0.91)= 0.31322\\ Effort = 87 PM\\ Duration = 3.67 * 87 \^ 0.31322 = 14.86 months\end{tabular} \\ \hline
		\end{tabular}
	\end{table}
	This is the Schedule which we have estimated.
	
	
	
	
	
	
	
	
	
	
	
	
	
\end{document}