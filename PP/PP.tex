\documentclass{article}
\usepackage{graphicx}
\usepackage[fontsize=11pt]{scrextend}
\title{PowerEnjoy Service - Project Plan}
\begin{document}

\begin{titlepage}
\begin{figure}
	\centering
	\includegraphics{polimi}
\end{figure}
\maketitle
\centering
Prof. Luca Mottola
\newline
\raggedleft
Authors:
\begin{itemize}
	\raggedleft
	\item ZHOU YINAN(Mat. 872686)
	\item ZHAO KAIXIN(Mat. 875464)
	\item ZHAN YUAN(Mat. 806508)	
\end{itemize}
\end{titlepage}

\tableofcontents
\newpage
\section{Introduction}
\subsection{Revision History}
Version 1.0
\subsection{Purpose and Scope}
This document aims at analyzing the overall complexity and making an estimation about the project size and required effort. The result will help project manager to decide the project budget, resource allocation and the schedule of activities.The document is divided into four parts. \\

In the first part of the document, We will use two specific methods to estimate the size and complexity of the project. First of all, we will use Function Points to calculate the average line of codes. Secondly, we will use COCOMO method to indicate the cost and effort estimation. \\

In the second part of the document, we will present the tasks for the project and the corresponding schedule. We will use the above results to come up with a suitable working plan covering the entire project development.\\

In the third part of the document, we will assign each team member specific missions to tickle down the project.\\

finally, we are going to analyze the risk we may encounter during the development. By analyzing the risk and coming up with possible solutions, we'll minimize the possibility for failure.
\subsection{Definitions, Acronyms, Abbreviations}
	\subsubsection{Definitions}
	\begin{itemize}
		\item Precedentedness: High if a product is similar to several previously developed projects
		\item Development Flexibility: High if there are no specific constraints to conform to pre-established requirements and external interface specs
		\item Architecture / Risk Resolution: High if we have a good risk management plan, clear definition of budget and schedule, focus on architectural definition
		\item Team Cohesion: High if all stakeholders are able to work in a team and share the same vision and commitment.
		\item Process Maturity: Refers to a well known method for assessing the maturity of a software organization, CMM, now evolved into CMMI 
	\end{itemize}
	\subsubsection{Acronyms , Abbreviations}
	\begin{itemize}
		\item FP : Function Point
		\item ILF : Internal Logic File
		\item ELF : External Logic File
		\item EI : External Input
		\item EO : External Output
		\item EQ : External Inquires
		\item PREC : Precedentedness
		\item FLEX : Development Flexibility 
		\item RESL : Risk Resolution
		\item TEAM : Team Cohesion
		\item PMAT : Process Maturity
	\end{itemize}
\subsection{Reference Documents}
\begin{itemize}
	\item Specification Document Assignments AA 2016-2017
	\item Function Point tables
	\item COCOMO tables
\end{itemize}
\newpage

\section{Project size, cost and effort estimation}
\subsection{Size estimation: function points}
	\subsubsection{Internal Logic Files(ILFs)}
	\subsubsection{External Logic Files(ELFs)}
	\subsubsection{External Inputs(EIs)}
	\subsubsection{External Inquires(EQs)}
	\subsubsection{External Outputs(EOs)}
	\subsubsection{Overal estinamtion}
\subsection{Cost and effort estimation: COCOMO II}
	\subsubsection{Scale Drivers}
	\subsubsection{Cost Drivers}
	\subsubsection{Effort equations}
	\subsubsection{Schedule estimation}
\newpage

\section{Schedule}

\newpage

\section{Resource allocation}

\newpage

\section{Risk management}

\newpage

\section{Effort}
\end{document}