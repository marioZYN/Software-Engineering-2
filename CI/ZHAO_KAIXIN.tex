\documentclass{article}
\usepackage{graphicx}
\usepackage[fontsize=11pt]{scrextend}
\title{Code Inspection}

\begin{document}

\section{ProductionRun.java}

\subsubsection{Naming Conventions}
\begin{enumerate}
\item
All class names, interface names, method names, class variables, method variable, and constant used are meaningful.

\item
The only one one-character variable in this ProductionRun.java file is \newline
GenericEntityException e \newline
And it is used as a parameter for the exception, for catching the statement. Therefore it is used in the loop, and it is temporary "throwaway" variable.

\item
Class name of this ProductionRun.java file is: \newline
ProductionRun\newline
Therefore, it is with the first letter of each word in capitalized.

\item
There is no ant defined interface within the ProductionRun.java file.

\item
The methods of this file are as follow:\newline
exist()\newline
getGenericValue()\newline
store()\newline
getProductionProduced()\newline
getQuantity()\newline
setQuantity()\newline
getEstimatedStartDate()\newline
setEstimatedStartDate()\newline
getEstimatedCompletionDate()\newline
setEstimatedCompletionDate()\newline
recalculateEstimatedCompletionDate()\newline
getProductionRunName()\newline
setProductionRunName()\newline
getDescription()\newline
setDescription()\newline
getCurrentStatus()\newline
getProductionRunComponents()\newline
getProductionRunRoutingTasks()\newline
getLastProductionRunRoutingTask()\newline
clearRoutingTasksList()\newline
getEstimatedTaskTime()\newline
isUpdateCompletionDate()\newline
All the names of methods are verbs, and with the first letter of each addition work capitalized.

\item
All of the class variables are as follow:\newline
productionRun\newline
productionRunProduct\newline
productionProduced\newline
quantity\newline
estimatedStartDate\newline
estimatedCompletionDate\newline
productionRunName\newline
description\newline
currentStatus\newline
productionRunRoutingTasks\newline
dispatcher\newline
There is no class variable with underscore, but all of the variables are lowercase first letter, and others with first letter capitalized.

\item
There are two constant:\newline
\begin{table}[hpb]
	\label{my-label}
	\begin{tabular}{|l|}
		\hline
		public static final String module 
		= ProductionRun.class.getName();\newline\\
		\hline
	\end{tabular}
\end{table} 
\begin{table}[hpb]
	\label{my-label}
	\begin{tabular}{|l|}
		\hline
		public static final String resource 
		= "ManufacturingUiLabels";\newline\\
		\hline
	\end{tabular}
\end{table} 
But "module" and "resource" should be modified to "MODULE" and "RESOURCE".
\end{enumerate}

\subsubsection{Indention}
For all indention, in the file ProductionRun.java, it adopts the convention of four space and done so consistently.\newline
For indention, there is no tab used.
 
\subsubsection{Braces}
Bracing style adopted for entire class is ”Kernighan and Ritchie” style.\newline 
For all body of all if-else,while,do-while,try-catch and for, the curly braces are used also for only one statement.

\subsubsection{File Organization}
\begin{enumerate}
	\item In this file, for all of the sections, there is a blank line to separate from each others.
	\item In this file, there is few line exceed 80 characters.
	\item In this file, there is no line exceed 120 characters.
\end{enumerate}

\subsubsection{Wrapping Line}
Every expressions in the ProducationRun.java file fit on a single line, so the convention is valid.

\subsubsection{Comments}
In this file, all of the comments are used to adequately explain what the class, interface, methods, and blocks of code are doing.
What is more, there are also some comments in the method, in order to explain the detail.

\subsubsection{Java Source Files}
\begin{enumerate}
	\item In this file, contains only one single public class. \newline
	\item In this file, this public class is this first class in the file. \newline
	\item There is no external program interface. \newline
	\item The javadoc is completed.
\end{enumerate}

\subsubsection{Package and Import Statements}
The package statements are in the first non-comment statement.\newline
And the Import statements follow with the package statements.

\subsubsection{Class and Interface Declarations}
The class declarations are in the correct order. And there is no interface.\newline
Which is:\newline
\begin{enumerate}
	\item class documentation comment.
	\item class statement.
	\item class (static) variable.
	\item instance variable.
	\item constructors.
	\item methods.
\end{enumerate}
The methods are grouped by the functionality.\newline

\newpage

\section{ViewerServletRequest.java}

\subsubsection{Initialization and Declaration}
\begin{enumerate}
\item 
All variables and class members are declared with correct type and the right visibility.
\item
All variables are declared in the proper scope.
\item
There is no call for the constructor.
\item
There is no object references which is used.
\item 
All variables are initialized where they are declared.
\item
All declarations appear at the beginning of block, some exceptions are declared after some instructions.
\end{enumerate}

\subsubsection{Method Calls}
\begin{enumerate}
	\item All of the parameters are presented in the correct order.
	\item There is only one method: getParameter(String name), and it is been call correctly.
	\item All method return type is correct
\end{enumerate}
 
\subsubsection{Arrays}
There is no array used in this file.

\subsubsection{Object Comparison}
There are only two comparison in this ViewerServletRequest.java file, and they are used correctly.

\subsubsection{Output Format}
The class return always the desired output.
The error message is managed in the classes of exception, so from this class we can not argue on the comprehensiveness.

\subsubsection{Computation, Comparisons and Assignments}
\begin{enumerate}
	\item In this ViewerServletRequest.java file, we do not have long and complex arithmetic expressions.
	\item All of the comparison and Boolean operators are correct.
	\item For the throw-catch expression, the error condition is actually legitimate
	\item The code does not contain any explicit and implicit type conversions.
\end{enumerate}

\subsection{time work}
6h checklist

\end{document}
