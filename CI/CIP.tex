\documentclass{article}
\usepackage{graphicx}
\usepackage[fontsize=11pt]{scrextend}
\usepackage{hyperref}
\title{Code Inspection}
\begin{document}

\begin{titlepage}
\begin{figure}
	\centering
	\includegraphics{polimi}
\end{figure}
\maketitle
\centering
Prof. Luca Mottola
\newline
\raggedleft
Authors:
\begin{itemize}
	\raggedleft
	\item ZHOU YINAN(Mat. 872686)
	\item ZHAO KAIXIN(Mat. 875464)
	\item ZHAN YUAN(Mat. 806508)	
\end{itemize}
\end{titlepage}

\tableofcontents
\newpage

\section{Classes Assigned}
We have been assigned two classes :
\begin{itemize}
	\item ProductionRun.java 
	\item ViewerServletRequest.java
\end{itemize}
The namespace patten is :
\begin{table}[h]
	\centering
	\label{my-label}
	\begin{tabular}{|l|}
		\hline
		../apache-ofbiz-16.11.01/applications/manufacturing/src/main/java/org/apache/ofbiz\\/manufacturing/jobshopmgt/ProductionRun.java \\ \hline
	\end{tabular}
\end{table}

\begin{table}[h]
	\centering

	\label{my-label}
	\begin{tabular}{|l|}
		\hline
		../apache-ofbiz-16.11.01/specialpurpose/birt/src/main/java/org/apache/ofbiz/birt\\/report/servlet/ViewerServletRequest.java \\ \hline
	\end{tabular}
\end{table}
\newpage
\section{Functional Role}
 \subsection{ProductionRun.java}
 Instead of directly looking into the code, we first examine the online ofbiz document to get information of this class. This class belongs to \textbf{Manufacturing} section. 
 The link to the 
 \href{https://cwiki.apache.org/confluence/display/OFBIZ/Beginner%27s+Guide+to+the+Manufacturing+Process}{Online Documet}. \\
\\The situation is described here. After a client makes order, configurable goods which our company provide require some type of manufacturing or production. If we do not have the requiring parts in our inventory, a production run is generated.

\begin{figure}[h]
	\centering
	\includegraphics[width=\textwidth]{web}
\end{figure}
If we log into the ofbiz web application, we can examine the production run section.
\begin{figure}[h]
	\centering
	\includegraphics[width=\textwidth]{productionRun}
\end{figure}
The ProductionRun class manages all the information of a certain production run activity. 
\newpage
 \subsection{ViewerServeltRequest}
 By looking at the code, we find that ViewerservletRequest extends HttpServletRequestWrapper. A "HttpServletRequestWrapper" provides a convenient implementation of the HttpServletRequest interface that can be subclassed by developers wishing to adapt the request to a Servlet. Thus the  role of this class is to represent a specific function of HttpServletRequest. More specifically, this function is getParameter();\\
 \\Before looking into this function, let's recall what is a servlet. A servlet lives in a web container, and it is responsible for generating dynamic web contents. Servlet can be viewed as a special java class without main methods. After a client sends a HTTP request to the web server, the web container is responsible for :
 \begin{itemize}
 	\item create an instance of a servlet
 	\item call specific method of a servlet
 	\item destroy a servlet
 \end{itemize}
\begin{figure}[h]
	\centering
	\includegraphics[width=\textwidth]{web_container}
\end{figure}

\begin{figure}[h]
\centering
\includegraphics[width=\textwidth]{handle_request}
\end{figure}
\newpage
The web container knows which servlet to call because a servlet can have three names :
\begin{itemize}
	\item Client knows URL name
	\item Deployer knows servlet secret internal name
	\item Actual java class name
\end{itemize}
 The XML document is responsible for deployment.
 \begin{figure}[h]
 	\centering
 	\includegraphics[scale = 0.7]{xml}
 \end{figure}
\newpage
Now let's look at what function role is this class. \textbf{ServletRequest} defines an object to provide client request information to a servlet. The servlet container creates a ServletRequest object and passes it as an argument to the servlet's service method. A \textbf{ServletRequest} object provides data including parameter name and values, attributes, and an input stream.\\
This java class file is used to form the parameter.
\section{Check List}
 \subsection{ProductionRun.java}
  \subsubsection{Naming Conventions}
  \begin{enumerate}
  	\item
  	All class names, interface names, method names, class variables, method variable, and constant used are meaningful.
  	
  	\item
  	The only one one-character variable in this ProductionRun.java file is \newline
  	GenericEntityException e \newline
  	And it is used as a parameter for the exception, for catching the statement. Therefore it is used in the loop, and it is temporary "throwaway" variable.
  	
  	\item
  	Class name of this ProductionRun.java file is: \newline
  	ProductionRun\newline
  	Therefore, it is with the first letter of each word in capitalized.
  	
  	\item
  	There is no ant defined interface within the ProductionRun.java file.
  	
  	\item
  	The methods of this file are as follow:\newline
  	exist()\newline
  	getGenericValue()\newline
  	store()\newline
  	getProductionProduced()\newline
  	getQuantity()\newline
  	setQuantity()\newline
  	getEstimatedStartDate()\newline
  	setEstimatedStartDate()\newline
  	getEstimatedCompletionDate()\newline
  	setEstimatedCompletionDate()\newline
  	recalculateEstimatedCompletionDate()\newline
  	getProductionRunName()\newline
  	setProductionRunName()\newline
  	getDescription()\newline
  	setDescription()\newline
  	getCurrentStatus()\newline
  	getProductionRunComponents()\newline
  	getProductionRunRoutingTasks()\newline
  	getLastProductionRunRoutingTask()\newline
  	clearRoutingTasksList()\newline
  	getEstimatedTaskTime()\newline
  	isUpdateCompletionDate()\newline
  	All the names of methods are verbs, and with the first letter of each addition work capitalized.
  	
  	\item
  	All of the class variables are as follow:\newline
  	productionRun\newline
  	productionRunProduct\newline
  	productionProduced\newline
  	quantity\newline
  	estimatedStartDate\newline
  	estimatedCompletionDate\newline
  	productionRunName\newline
  	description\newline
  	currentStatus\newline
  	productionRunRoutingTasks\newline
  	dispatcher\newline
  	There is no class variable with underscore, but all of the variables are lowercase first letter, and others with first letter capitalized.
  	
  	\item
  	There are two constant:\newline
  	\begin{table}[hpb]
  		\label{my-label}
  		\begin{tabular}{|l|}
  			\hline
  			public static final String module 
  			= ProductionRun.class.getName();\newline\\
  			\hline
  		\end{tabular}
  	\end{table} 
  	\begin{table}[hpb]
  		\label{my-label}
  		\begin{tabular}{|l|}
  			\hline
  			public static final String resource 
  			= "ManufacturingUiLabels";\newline\\
  			\hline
  		\end{tabular}
  	\end{table} 
  	But "module" and "resource" should be modified to "MODULE" and "RESOURCE".
  \end{enumerate}
  
  \subsubsection{Indention}
  For all indention, in the file ProductionRun.java, it adopts the convention of four space and done so consistently.\newline
  For indention, there is no tab used.
  
  \subsubsection{Braces}
  Bracing style adopted for entire class is ”Kernighan and Ritchie” style.\newline 
  For all body of all if-else,while,do-while,try-catch and for, the curly braces are used also for only one statement.
  
  \subsubsection{File Organization}
  \begin{enumerate}
  	\item In this file, for all of the sections, there is a blank line to separate from each others.
  	\item In this file, there is few line exceed 80 characters.
  	\item In this file, there is no line exceed 120 characters.
  \end{enumerate}
  
  \subsubsection{Wrapping Line}
  Every expressions in the ProducationRun.java file fit on a single line, so the convention is valid.
  
  \subsubsection{Comments}
  In this file, all of the comments are used to adequately explain what the class, interface, methods, and blocks of code are doing.
  What is more, there are also some comments in the method, in order to explain the detail.
  
  \subsubsection{Java Source Files}
  \begin{enumerate}
  	\item In this file, contains only one single public class. \newline
  	\item In this file, this public class is this first class in the file. \newline
  	\item There is no external program interface. \newline
  	\item The javadoc is completed.
  \end{enumerate}
  
  \subsubsection{Package and Import Statements}
  The package statements are in the first non-comment statement.\newline
  And the Import statements follow with the package statements.
  
  \subsubsection{Class and Interface Declarations}
  The class declarations are in the correct order. And there is no interface.\newline
  Which is:\newline
  \begin{enumerate}
  	\item class documentation comment.
  	\item class statement.
  	\item class (static) variable.
  	\item instance variable.
  	\item constructors.
  	\item methods.
  \end{enumerate}
  The methods are grouped by the functionality.\newline
  
\begin{enumerate}
	\item
	All variable and class members are declared with correct type and the right visibility.
	\item
	All variables are being used only in the scope where they are declared.
	\item
	We do not have a constructor with  an empty parameter, so when declaring there is no  default constructor called. But when the class ProductionRun is initialized, there is a constructor that can be called.
	\item
	All object references are initialized before used.
	\item
	Several variable attributes have not been initialized explicitly. They may assume a standard value in phase of computation. 
	\item
	Almost all declarations appear at the beginning of block, except some are declared after some instructions.
\end{enumerate}
\subsubsection{Method Calls}
All parameters are presented in the correct order.\\
We have found two pairs of method that have same name: getEstimatedTaskTime and recalculateEstimatedCompletionDate, but each of them  refers to the same functionality.\\
All method return type is correct.
\subsubsection{Array}
There are no problem with off-by-one error or out-of-bounds, we manager the array using iterator instead of index.
\subsubsection{Object Comparison}
In class is always used == to compare a object with NULL, and just one use equals (line 85).
\subsubsection{Output Format}
The class return always the desired output.\\
The error message is managed in the classes of exception, so from this class we can not argue on the comprehensiveness.
\subsubsection{Computation, Comparisons and Assignments}
In this java class we do not have long and complex arithmetic expressions. And there is not special arithmetic expression to be taken with particular attention(like division), therefore, there are no operator precedence problems.\\
Other operators are also in correct form.\\
The code does not contain any explicit and implicit type conversions.
\subsubsection{Exceptions}
For every try statement there are at least one catch statement that take care of exceptions.
\subsubsection{Flow of Control}
In the class there are not any switch statement. And for loops,  they are correct.
\subsubsection{Files}
This class does not manage the files.

\newpage
 \subsection{ViewerServletRequest}
\subsubsection{Naming Conventions}
\begin{enumerate}
	\item 
	The name for class ViewerServletRequest.java, and all name of its attributes, method and constant are meaningful.
	\item
	There is only one one-character variable, and it is used as parameter of catch statement, therefore it is "throwaway" variable.
	\item
	The class name is composed with three nouns, initial letter of each word is capitalized.
	\item
	We have not defined any interface within the class.
	\item 
	There is only one method in class: getParameter(), it contains a verb, also, every addition word begin with capitalized letter.
	\item
	We do not have attributes beginning with an underscore, whatever, the initial word is lowercase, and first letter of each others is capitalized.
	\item
	In class has a constant is written in lowercase:
	\begin{table}[hpb]
		\label{my-label}
		\begin{tabular}{|l|}
			\hline
			public final static String module = ViewerServletRequest.class.getName();\\
			\hline
		\end{tabular}
	\end{table} 
	\\Where 'module' should be written using all uppercase.
\end{enumerate}

\subsubsection{Indention}
For all indentation, we adopt the convention of four space. There are not tab used to indent. 
\subsubsection{Braces}
Bracing style adopted for entire class is "Kernighan and Ritchie" style. For all body of all if-else,while,do-while,try-catch and for, the curly braces are used also for only one statement.
\subsubsection{File Organization}
\begin{enumerate}
	\item
	For each section there is a blank line to separate from others.
	\item
	There only few line exceed 80 character (38,48,54 and 59). Neither of them exceed 120 characters.
\end{enumerate}
\subsubsection{Wrapping Lines}
Every expression in the class fit on a single line, so the convention is valid.
\subsubsection{Comments}
The class is completely lack of comment.
 \subsubsection{Java Source Files}
 There is external program interface, HTTPServletRequest. 
 \subsubsection{Package and Import Statements}
 The package statements are in the first non-comment statement. And the Import statements follow with the package statements.
 \subsubsection{Class and Interface Declarations}
 No problem.
\subsubsection{Initialization and Declaration}
\begin{enumerate}
	\item 
	All variables and class members are declared with correct type and the right visibility.
	\item
	All variables are declared in the proper scope.
	\item
	There is no call for the constructor.
	\item
	There is no object references which is used.
	\item 
	All variables are initialized where they are declared.
	\item
	All declarations appear at the beginning of block, some exceptions are declared after some instructions.
\end{enumerate}

\subsubsection{Method Calls}
\begin{enumerate}
	\item All of the parameters are presented in the correct order.
	\item There is only one method: getParameter(String name), and it is been call correctly.
	\item All method return type is correct
\end{enumerate}

\subsubsection{Arrays}
There is no array used in this file.

\subsubsection{Object Comparison}
There are only two comparison in this ViewerServletRequest.java file, and they are used correctly.

\subsubsection{Output Format}
The class return always the desired output.
The error message is managed in the classes of exception, so from this class we can not argue on the comprehensiveness.

\subsubsection{Computation, Comparisons and Assignments}
\begin{enumerate}
	\item In this ViewerServletRequest.java file, we do not have long and complex arithmetic expressions.
	\item All of the comparison and Boolean operators are correct.
	\item For the throw-catch expression, the error condition is actually legitimate
	\item The code does not contain any explicit and implicit type conversions.
\end{enumerate}
 \subsubsection{Exceptions}
 1. The relevant exception is caught. 
 \begin{verbatim}
 try {
 reportFileUrl = FlexibleLocation.resolveLocation(reportParam, loader);
 } catch (MalformedURLException e) {
 Debug.logError(e, module);
 }
 if (reportFileUrl == null) {
 throw new IllegalArgumentException("Could not resolve location to URL: " 
 + reportParam);
 }
 \end{verbatim}
 The function in this code block tries to locate the file url. In case of wrong url and no file found, an exception is raised.
 \subsubsection{Flow of Control}
 No \textbf{switch} and \textbf{loop} in this file.
 \subsubsection{Files}
 This java class does not deal with file operations. 
 
 \section{Work Time}
 \begin{itemize}
 	\item 21/01/2017 ZHOU YINAN 1h document structure
 	\item 02/02/2017 ZHOU YINAN 3h function role and checklist
 	\item 02/02/2017 ZHAN YUAN	4.5h checklist
 	\item 04/02/2017 ZHAO KAIXIN 6h checklist
 \end{itemize}
\end{document}